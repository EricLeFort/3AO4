\documentclass[titlepage]{article}
\usepackage[left=15mm,right=15mm,top=1in,bottom=1in]{geometry}
\usepackage{graphicx}
\graphicspath{ {C:\Users\Nick\Documents\GitHub\3A04\3AO4\doc\intermediateFiles} }

\title{Stone Identification App \\
	Software Requirements Specification}
\author{Genevieve Okon (okong), Sydney Lieng(liengsn),\\
	Niko Savas(savasn), Nick Lago(lagond),\\
	Eric Le Fort(leforte)}
\date{\today}

\begin{document}
\maketitle
\newpage

\section{Non-Functional Requirements}

The following section will outline the various non-functional requirements that pertain to the Rock Identification app.

\subsection{Look \& Feel Requirements}
These requirements dictate how the final product will appear to the user. The overall goal will be to make the app's look \& feel in a way that promotes simple and efficient use by the average user.
\subsubsection{Appearance Requirements}
\textbf{Ap1:} The program shall use colour schemes appealing to the expected users determined by the designers.\\

\noindent\textbf{Ap2:} The program's appearance shall be designed in a way that it does not distract from usage of the system.\\

\noindent\textbf{Ap3:} The program shall be easy to see even in environments with high background light (i.e. outdoors).

\subsubsection{Style Requirements}
\textbf{S1:} The program shall uphold a simple design with each page displaying only the necessary information.

\subsection{Usability \& Humanity Requirements}
\subsubsection{Ease of Use Requirements}
\textbf{EOF1:} The program shall be usable by any user between the ages of 8 and 90 with basic knowledge of how apps function.
\subsubsection{Personalization \& Internalization Requirements}
\textbf{PI1:} The program shall allow the user to choose whether they would like to use location services.
\subsubsection{Learning Requirements}
\textbf{Le1:} The user shall require training in order to understand usage of basic qualifying characteristics of rocks.
\subsubsection{Understandability \& Politeness Requirements}
\textbf{UP1:} The program shall only assume prior knowledge of the basics of how to use an app and simple defining characteristics of rocks. Functions that require knowledge other than this will not be included.

\noindent\textbf{UP2:} The program shall include information in \textit{Layman's Terms} to support the more technical information.
\subsubsection{Accessibility Requirements}
\textbf{Ac1:} The program shall utilize the largest text possible while maintaining it's aesthetic appeal.\\

\subsection{Performance Requirements}
\subsubsection{Speed \& Latency Requirements}
\textbf{SL1:} Time computing possible matches and displaying those results shall never exceed 1 second.\\

\noindent\textbf{SL2:} Time to start the program shall never exceed 3 seconds.
\subsubsection{Safety Critical Requirements}
N/A
\subsubsection{Precision \& Accuracy Requirements}
\textbf{PA1:} The program shall display the correct rock that matches the descriptors (assuming they are correct) 95\% of the time.\\

\noindent\textbf{PA2:} The program shall display the correct rock that matches the descriptors (assuming small mistakes by user being possible) 80\% of the time.
\subsubsection{Reliability and Availability Requirements}
\textbf{RA1:} This system is available on the user's machine and does not rely on other factors to function. This program should be available 99\% of the time.\\

\noindent\textbf{RA2:} The program shall \textit{crash} no more than an average of once per hour of runtime.
\subsubsection{Robustness or Fault-Tolerance Requirements}
\textbf{RFT1:} The program shall be designed in a way that faulty inputs are not possible to select.
\subsubsection{Capacity Requirements}
No requirements necessary from this section since the program will run independently on each piece of hardware it is running on. So the maximum capacity is only 1.
\subsubsection{Scalability or Extensibility Requirements}
\textbf{SE2:} The program shall be designed in a way that adding new rocks to the database is possible after deployment without major changes.
\subsubsection{Longevity Requirements}
\textbf{Lo1:} The product shall be designed in a modular way so as to allow future anticipated change.

\subsection{Operational \& Environmental Requirements}
\subsubsection{Expected Physical Environment}
\textbf{EP1:} The program will primarily be operated outdoors. The program shall be designed such that non-extreme environments will not affect usability or performance.
\subsubsection{Requirements for Interfacing with Adjacent Systems}
\textbf{RIA1:} The program shall interface with Android Jelly Bean (4.1.x) and higher.
\subsubsection{Productization Requirements}
\textbf{Pro1:} The program shall come packaged with all required libraries necessary to readily interface with Android Jelly Bean (4.1.x) and higher.
\subsubsection{Release Requirements}
\textbf{R1:} The program shall be ready for release of version 1.0 by April $3^{rd}$, 2016.

\subsection{Maintainability \& Support Requirements}
\subsubsection{Maintenance Requirements}
\textbf{MS1:} The product shall store its data in such a way that it will be able to restore to a previous state as necessary.
\subsubsection{Supportability Requirements}
\textbf{Su1:} The program shall follow a modular design pattern. No more than 10\% of modules will be so rigid they can only operate in the exact implementation of the current system.
\subsubsection{Adaptability Requirements}
\textbf{Ad1:} The program shall be able to integrate new types of rocks into its identification system without affecting any other component's operation.

\subsection{Security Requirements}
\subsubsection{Access Requirements}
\textbf{AR1:} Accessing the identification feature will not require credentials, however, accessing the previous rocks discovered or adding new rocks will require the user entering their password.
\subsubsection{Integrity Requirements}
\textbf{In1:} The rock data stored locally will be encrypted and compared against fixed values as needed to verify there has not been any tampering.\\

\noindent\textbf{In2:} The user's password will be padded and stored encrypted with two different methods of encoding. Both methods of encoding must return the same password to ensure there has not been any tampering.
\subsubsection{Privacy Requirements}
\textbf{Pri1:} The program will not disclose any information about the user to third-parties without permission.
\subsubsection{Audit Requirements}
N/A
\subsubsection{Immunity Requirements}
\textbf{Im1:} The program shall make all files that do not change due to normal program operation \textit{read-only} in the file system.\\

\noindent\textbf{Im2:} The program shall make an announcement to the user if any files hold unexpected data.

\subsection{Cultural \& Political Requirements}
\subsubsection{Cultural Requirements}
\textbf{Cu1:} This program shall not include any symbols, statements or anything of that nature that may be found offensive to anyone's culture.
\subsubsection{Political Requirements}
\textbf{Po1:} This program shall not include any symbols, statements or anything of that nature that are offensive in any region in a political sense.

\subsection{Legal Requirements}
\subsubsection{Compliance Requirements}
\textbf{C1:} The program shall not retain sensitive data about the user without the user's permission.
\subsubsection{Standards Requirements}
N/A
\end{document}