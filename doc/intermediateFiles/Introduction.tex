\documentclass[titlepage]{article}
\usepackage[left=15mm,right=15mm,top=1in,bottom=1in]{geometry}
\begin{document}
	\title{Stone Identification App \\
		Software Requirements Specification}
	\author{Genevieve Okon (Okong), Sydney Lieng(liengsn),\\
		Niko Savas(), Nick Lago(lagond),\\
		Eric Le Fort(leforte)}
	\date{\today}
	\maketitle
	
\section{Introduction}
\subsection{Purpose}
The following Software Requirements Specification (SRS) document will provide an overview of the behaviour of the Stone Identification App that will be developed. This document will include a full description of the application and will outline the functional and non-functional requirements that the system needs to fulfill. To gain a thorough understanding of this application, this SRS is geared towards those who are interested in developing and funding this application. It will give them insight to what the software will do and how it will be expected to perform.
\subsection{Scope}
The Stone Identification App will assist users with the classification of rocks. The system will be able to take in descriptions, such as colour, texture, and size, to recognize the type of rock. When searching through the database, the application will display images of the possible matches, a description, and how much it will cost per gram. In addition, the application will access Google maps API to help narrow the search by location, as well as show others in the area who have also have found similar rocks. Users will also be able to search for rock names and see a description, value and images of the selected rock. The goals of this application is to allow those who are interested in rocks to have a quick access to a database and search through its resources. This application is also used to encourage those who do not know much about rocks to learn more about their qualities. 
\subsection{Definitions, Acronyms \& Abbreviations}
\textbf{Software Requirements Specification (SRS):} a complete description of the behaviour of a system to be developed. It includes a set of use cases that describe interaction betweeen the system and the user/environment.\\
\textbf{Layman's Terms:} describe a complex or technical issue using words and terms that the average individual (someone without professional training in the subject area) can understand, so that they may comprehend the issue to some degree.\\
\textbf{Crash:} when a software stops functioning properly\\
\textbf{Read-only:} cannot write or change the file's contents
\subsection{References}
\textbf{Andrew LeClair.}\\
SE 3A04: Requirements Templates\\
\textit{Department of Computing and Software, McMaster University, January 27/28, 2016}
\subsection{Overview}
The remaining SRS document will include the overall description, functional requirements, and non-functional requirements. It will provide a description of the general factors that affect the product and its requirements in the Overall Description section. The Functional Requirements section will contain all the software requirements to a level of detail. The last section, Non-Functional Requirements, will outline the how the system is supposed to be, such as the look and feel, usability and humanity, performance, etc. 

\end{document}